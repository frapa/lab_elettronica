\section*{Dati e risultati}

Per ogni resistenza a nostra disposizione abbiamo preso una misura volt-amperometrica per ogni circuito a nostra disposizione. Pertanto per ogni resistenza abbiamo due coppie di valori di corrente e tensione $(I\ped{x}, V\ped{x})$.
Quindi grazie ai valori acquisiti possiamo determinare il valore sperimentale delle resistenze sempre sfruttando la legge integrale di Ohm, grazie al fatto che le resistenze sono componenti circuitali Ohmici.

\begin{equation}
	R\ped{x} \,=\, \frac{V\ped{x}}{I\ped{x}}
\end{equation}

Inoltre per avere una verifica dei dati ottenuti abbiamo misurato il valore delle varie resistenze mediante il multimetro digitale.
Pertanto di seguito sono riportati tutti i valori delle resistenze ottenuti sperimentalmente e grazie al multimetro digitale. In questo modo potremo osservare se ci sono delle discrepanze importanti tra il valore teorico e quello sperimentale. 

