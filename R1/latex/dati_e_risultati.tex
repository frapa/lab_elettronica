\section*{Dati e risultati}

Per ogni resistenza a nostra disposizione abbiamo effettuato una misura volt-amperometrica con entrambi i circuiti realizzati. Pertanto per ognuna delle resistenze abbiamo due coppie di valori di corrente e tensione $(I\ped{x}, V\ped{x})$.
Quindi grazie ai valori acquisiti possiamo determinare il valore sperimentale delle resistenze, sempre sfruttando la legge integrale di Ohm, poichè le resistenze sono componenti circuitali Ohmici.

\begin{equation}
	R\ped{x} \,=\, \frac{V\ped{x}}{I\ped{x}}
\end{equation}

Inoltre per avere una verifica dei dati ottenuti abbiamo misurato il valore delle varie resistenze mediante il multimetro digitale.

Di seguito sono riportati tutti i valori delle resistenze ottenuti sia sperimentalmente  che con il multimetro digitale. In questo modo potremo osservare se ci sono delle discrepanze rilevanti tra il valore teorico e quello sperimentale delle resistenze. 

[GRAFICO]

Infine abbiamo misurato il valore della resistenza di una lampadina, in grado di sopportate una differenza di potenziale massima di $12\,\si{\volt}$, prendendo una serie di misure volt-amperometriche utilizzando entrambi i circuiti a nostra disposizione.
Anche in questo caso abbiamo graficato i dati e abbiamo ottenuto il seguente risultato:

[GRAFICO]
