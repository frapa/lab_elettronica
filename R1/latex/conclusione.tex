\section*{Conclusione}

Come si può osservare dall'andamento del grafico relativo alla resistenza della lampadina, questo non presenta un'andamento lineare.
Questo fatto è particolarmente evidente sia per valori bassi di corrente elettrica sia dal fatto che l'intercetta del grafico non passa per l'origine.

Questo risultato era prevedibile dal momento che la resistenza in esame non è un componente Ohmico e quindi non segue la legge di Ohm.
O meglio la legge sarebbe rispettata nel caso in cui la temperatura della resistenza non variasse significatevamente per varie differenze d potenziale. Tuttavia come è ben noto una lampadina a incandescanza sfrutta l'effetto Joule per scaldarsi ed emettere radiazione luminosa. Per questo motivo la sua temperatura non è per nulla costante e quindi non può essere approssimata ad un componente Ohmico.
Inoltre possiamo osservare che per valori di corrente bassa l'andamento è visibilmente non lineare. Inoltre l'intercetta del grafico non passando per l'origine fornisce un'ulteriore prova che il componente non è Ohmico
