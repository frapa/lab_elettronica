\section*{Conclusione}

Come si può osservare dall'andamento del grafico inerente alla resistenza della lampadina, questo non presenta un'andamento lineare. Questo risultato era prevedibile dal momento che la resistenza in esame non è un componente Ohmico e quindi non segue la legge di Ohm. O meglio la legge sarebbe rispettata nel caso in cui la temperatura della resistenza non variasse significatevamente ,ma come sappiamo una lampadina a incandescanza sfrutta l'effetto Joule per scaldarsi ed emettere radiazione luminosa, quindi la sua temperatura non è per nulla costante.
Inoltre possiamo osservare che per valori di corrente bassa l'andamento è visibilmente non lineare. Inoltre l'intercetta del grafico non passando per l'origine fornisce un'ulteriore prova che il componente non è Ohmico
