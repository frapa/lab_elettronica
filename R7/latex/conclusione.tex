\section*{Conclusione}

Per quanto riguarda il rilevatore di picchi durante l'esperienza abbiamo osservato il suo corretto funzionamento sull'oscilloscopio. Inoltre analizzando il grafico in Fgura \ref{fig:capacita} possiamo notare che, fissato il valore di capacità $C$ ad $1\,\si{\micro\farad}$, si ottiene il seganle massimo in uscita ad una valore di resistenza $R_1$ di $1000\,\si{\ohm}$. Per valori ineriori di $R_1$ $V\ped{out}$ diminuisce, infatti se la resistenza è bassa parte del segnale si perde. Stessa cosa accade nel caso in cui $R_1$ è molto grande rispetto a $1\,\si{\kilo\ohm}$. Questo è dovuto al fatto che: Boh non lo ho capito!!!!.\\

Infine se analizziamo i risultati ottenuti per il diodo Zener, modello XZY85C10, possiamo dire che la tensione di breakdown è ad una valore di tensione di $-10\,\si{\volt}$ come si può notare dalla Figura \ref{fig:caratteristica_I-V}. Inoltre la sua resistenza dinamica $R_d$, nella regione di conduzione del diodo Zener, ha un valore di $14.2\,\pm\,0.1\,\,\si{\ohm}$. Quindi conoscendo questo valore siamo stati in grado di verificare i se i rapporti di stabilizzazione teorico ($\chi\ped{teo}$) e sperimentale ($\chi\ped{exp}$) risultino compatibili tra di loro. Quindi in base ai valori ottenuti dalle equazioi \ref{eq:c_exp} e \ref{eq:c_teo} possiamo dire che i due valori sono compatibili entro le loro incertezze.
Inoltre avvalendoci dei dati illustrati in Figura \ref{fig:stab_tensione} possiamo dire che il range di funzionamento del npstro circuito stabilizzatore di tensione va da $10\,\si{vot}$ fino a $25\,\si{\volt}$ che è il massimo valore di tensione in ingresso che possiamo raggiungere con il generatore di tensione in dotazione. 
