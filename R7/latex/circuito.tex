\section*{Circuito}

Per la costruzione dei circuiti e le misure abbiamo utilizzato i seguenti componenti:

\begin{itemize}
    \setlength{\itemsep}{1pt}
    \item{Breadboard, cavi a banana e cavetti per breadboard.}
    \item{Multimetro, generatore di forme d'onda, alimentatore di corrente continua e oscilloscopio.}
    \item{Una resistenza da \SI{1}{\kilo\ohm},
        un diodo 1N4007 e un diodo Zener XZY85C10.}
    \item{Decadi di resistenze e capacità.}
\end{itemize}

Nelle figure sottostanti sono riportati gli schemi dei due circuiti che abbiamo realizzato.

\begin{figure}[h]
  \centering
  \begin{subfigure}[b]{0.47\textwidth}
      \begin{circuitikz}[scale=0.8, transform shape, font=\Large] \draw
       (0,0)
        node[anchor=east]{$V\ped{in}$}
        to [vC, l=$C$, o-] (3, 0)
        to [D, l=$D$] (6, 0)
        to [short, -o] (9, 0)
        node[anchor=west]{$V\ped{out}$}
       (3, 0)
        to [vR, l=$R_1$] (3, -3)
        node[ground] {}
       (6, 0)
        to [R, l=$R_2$] (6, -3)
        node[ground] {}
        ;
      \end{circuitikz}
      \caption{Rilevatore di picchi}
      \label{fig:circuito_peak}
  \end{subfigure}
  \qquad \qquad
  \begin{subfigure}[b]{0.35\textwidth}
      \begin{circuitikz}[scale=0.8, transform shape, font=\Large] \draw
       (0,0)
        node[anchor=east]{$V\ped{in}$}
        to [R, l=$R$, o-] (3, 0)
        to [short, -o] (6, 0)
        node[anchor=west]{$V\ped{out}$}
       (3, -3)
        node[ground] {}
        to [zD, l=$D$] (3, 0)
        ;
      \end{circuitikz}
      \caption{Stabilizzatore di tensione}
      \label{fig:circuito_zener}
  \end{subfigure}
\end{figure}
