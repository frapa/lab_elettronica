\section*{Dati e risultati}

\subsection*{Diodo 1N4007}

Per studiare la caratteristica $I-V$ (corrente-tensione) di un diodo 1N4007 sia in polarizzazione inversa che in diretta ci siamo mossi come segue: abbiamo collegato il diodo all'alimentatore di tensione continua, misurando con il multimetro la corrente assorbita dal diodo.
Ricordiamo che durante questa prima fase dell'esperienza abbiamo prestato attenzione che la corrente attraversante il diodo non superasse i $700\,\si{\milli\ampere}$. Grazie ai dati misurati abbiamo graficato la curva caratteristica $I-V$ del diodo, mostrata in figura %\ref{fig:diodo}.

\subsection*{Cella fotovoltaica}

Come secondo obbiettivo abbiamo valutato la caratteristica $I-V$ di una cella fotovoltaica al silicio. Questo andamento è stato valutato sia ponendo la cella fotovoltaica al buio, ovvero ponendola all'interno del proprio invoucro di cartone, sia esponendola ad una sorgente luminosa.
Come nel caso precedente abbiamo misurato mediante il multimetro la correte passante nella cella fotovoltaica.
Anche in questo caso abbiamo dovuto fare attenzione che la corrente massima nella cellafotovoltaica non superasse i $100\,\si{\milli\ampere}$.

[GRAFICO]

Infine abbiamo calcolato il FillFactor, ovvero:::
