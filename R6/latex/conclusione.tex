\section*{Conclusione}

Per quanto riguarda lo studio del diodo 1N4007 possiamo dire che l'andamento intensità di corrente vs. tensione ($I-V$) risulta essere corretto. Infatti abbiamo un valore di tensione di cut-in di circa $0.5\,\si{\volt}$ che non è molto distante dal valore di tensione dei diodi al silicio che è di $0.6\,\si{\volt}$.
Inoltre come si può osservare dal grafico in Figura \ref{fig:diodo} non siamo riusciti a valutare la tensione di breakdown in quanto il range di funzionamento del nostro multimetro non era abastanza ampio, $(-50,\,+50)\,\si{\volt}$, per permetterci di ottenere la tensione sufficiente per visualizzare il fenomeno. Infatti sul datasheet del diodo in esame è riportato che la tensione di breakdown è di circa $1000\,\si{\volt}$.\\

Analizzando la cella fotovoltaica monocristallina al silicio posta al buio (Figura \ref{fig:buio}) possiamo notare che la curva caratteristica ($I-V$) risulta essere molto simile a quella del diodo precedentemente esaminato salvo il fatto che la tensione di cut-in risulta essere di circa $5\,\si{\volt}$.
Nel momento in cui la cella fotovoltaica è stata fatta funzionare esponendola ad una sorgente luminosa, l'andamento $I-V$ della stessa è sensibilmente variato, infatti come si può osservare dal grafigo in Figura \ref{fig:luce} anche quando alla cella non è applicata alcuna tensione vi è comunque una corrente di poco meno di $-20\,\si{\milli\ampere}$ che circola all'interno del circuito.
Inoltre una volta che la cella viene sottoposta ad una minima differenza di potenziale inizia quasi istantanemete a lavorare. Infatti se prendiamo un valore arbitrario di intensità di corrente, ad esempio $20\,\si{\milli\ampere}$ possiamo osservare come per la cella pota al buio servisse una tensione di circa $6\,\si{\volt}$, mentre se esposta ad una sorgente luminosa non occorre altro che una diffrenza di potenziale di circa $1\,\si{\volt}$.
Infine abbiamo ottenuto che il FillFactor vale TOT. Grazie a questo valore possiamo dire che la nostra cella fotovoltaica monocristallina al silicio è di qualità elevata.\\

Per quanto riguarda il ponte di Graetz i risultati da noi ottenuti sono tutti riportati in Figura \ref{fig:graetz2} e nella tabella in Tabella \ref{tab:ripple}. L'unica osservazione che aggiungiamo e quella di far notare come a frequenza assegnata di $100\,\si{\hertz}$ il fattore di ripple diventa sempre più piccolo man mano che la capacità aumenta. Questo quindi ci permette di dire che maggiore è la capacità applicata migliore sarà il livellamento del segnale in uscita. Infatti il circuito che abbiamo utilizzato è una semplificazione di un raddrizzatore a doppia semionda che in output da un segnale in corrente continua.