\section*{Dati e risultati}

Nella prima parte dell'esperienza abbiamo calcolato il tempo caratteristico ($\tau$) del nostro circuito $R\,C$ per tre valori differenti sia di $R$, la resistenza del circuito che di $C$, la capacità del condensatore. Per essere sicuri dei valori di resistenza e di capacità utilizzati gli abbiamo misurati grazie al multimetro digitale. Nel resto della relazione saranno usati questi valori per elaborare i dati e non quelli nominali.

Noi sappiamo che per un circuito $R\,C$, nel quale tutti i componenti sono ohmici, la relazione che sussiste tra il tempo caratteristico ($\tau$), la resistenza ($R$) e il condenzatore ($C$) è la seguente:

\begin{equation}
	\tau \,=\, R\,C
	\label{eq:tau}
\end{equation}
%
Per avere un riscontro con i dati ricavati sfruttando la relazione (\ref{eq:tau}) abbiamo utilizzato l'oscilloscopio, correttamente tarato, per ottenere il valore del tempo caratteristico del circuito, leggendo il valore del tempo ad una determinata tensione. Sapendo che in un circuito $R\,C$ la diffrenza di potenziale in funzione del tempo segue la legge:

\begin{equation}
	V\ped{t} \,=\, V\ped{0}\,(1\,-\,e^{-\frac{t}{\tau}})
	\label{eq:potenziale}
\end{equation}
%
se calcoliamo la differenza di potenziale al tempo $\tau$ otteniamo che la tensione al tempo caratteristico vale esattamente:

\begin{equation}
	V\ped{\tau} \,=\, 0.632\,V\ped{0} 
\end{equation}
%
dove in entrambe le equazioni $V\ped{0}$ indica la differenza di potenziale in uscita dal generatore di funzioni d'onda, $t$ è il tempo e $\tau$ è il tempo caratteristico del circuito.

Quindi nella seguente tabella sono riportati i valori del tempo caratteristico ricavato teoricamente ($\tau\ped{teo}$) risolvendo l'equazione (\ref{eq:tau}), e quello ricavato dalla lettura del multimetro ($\tau\ped{exp}$).

\begin{SCtable}[0.5][H]
  \centering
  \begin{tabular}{l | c c c}
      \multicolumn{4}{c}{\textbf{Tempo caratteristico $(\tau)$}} \\
      \toprule
      Resistenza $[\si{\ohm}]$ & Capacità $[\si{\farad}]$ & $\tau\ped{exp} [\si{\second}]$ & $\tau\ped{teo} [\si{\second}]$ \\
      \midrule
      499k   &  $5.00\,10^{-9}$ 	& $2.40\,10^{-3}$ 	& $2.498\,10^{-3}$ \\
      19.9k  &  $0.99\,10^{-6}$ 	& $2.08\,10^{-3}$ 	& $1.980\,10^{-3}$ \\
      997    &  $99.3\,10^{-9}$ 	& $1.04\,10^{-3}$ 	& $0.987\,10^{-3}$ \\
      \bottomrule
  \end{tabular}
  \caption{In questa tabella, come anticipato, sono riportati i valori delle resistenze vere, misurate mediante il multimetro, e quelli ricavati risolvendo i due circuiti a nostra disposizione. Come si vede chiaramente per resistenze con valore elevato abbiamo che la configurazione con amperometro a valle risulta essere più precisa di quella a monte. Il discorso è analogo se si considera la confuigurazione a monte, che risulta essere migliore per misurare resistenze piccole.}
\end{SCtable}

Infine abbiamo calcolato il valore della capacità, a priori incognita, di un condensatore. Questo è stato fatto risolvendo la semplice equazione:

\begin{equation}
	C \,=\, \frac{\tau}{R}
\end{equation}
%
dove i valori di $R$ e $\tau$ sono noti in quanto la resistenza è stata impostata a piacere e il tempo carateristico acquisito sperimentalmente grazie all'oscilloscopio.
