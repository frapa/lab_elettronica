\section*{Conclusione}

La capacità incognita che troviamo risolvendo l'equazione (\ref{eq:cap}) è risultata valere:

\begin{equation}
  C \,=\, 1.0 \pm 0.1 \: \si{\micro\farad}
\end{equation}
%
valore che poi è stato confermato dalla misura dello stesso col multimetro digitale.

Infine se osserviamo i risultati ottenuti per le misure dei tempi caratteristici del circuito $RC$, riportati in tabela (\ref{tab:tris}), i tempi caratteristici teorici sono compatibili con quelli misurati sperimentalmente entro le loro incertezze.
Questo risultato non ci sorprende dal momento che tutte le relazioni utilizzate sono state ricavate grazie alle leggi di Ohm, che ricordiamo sono valide nel caso in cui gli elementi del circuito siano Ohmici. La relazione della tensione in funzione del tempo è stata ricavata sfruttando le leggi di Kirchhoff e risolvendo la successiva equazone differenziale.