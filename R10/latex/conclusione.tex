\section*{Conclusione}

\subsection*{Amplificatore alle diffrenze}

Abbiamo verificato che il circuito \ref{fig:semplice} è effettivamente un amplificatore differenziale. Abbiamo quindi calcolato il guadagno e il CMRR, che è un
indicatore della qualità dell'amplificatore, trovando valori coerenti con quanto si può leggere sui manuali di elettronica. Inoltre abbiamo verificato le debolezze ed i limiti
di questo tipo di amplificatore, in particolar modo il problema dell'amplificazione in modo comune. C'è comunque da tenere a mente che, nonostante questo difetto,
l'amplificatore è molto stabile termicamente, essendo costituito da due transitor simili che hanno risposte quasi uguali ai cambiamenti di temperatura. Questa simmetria
permette di cancellare gli effetti sui singoli transitor, che essendo dispositivi a semiconduttore sono influenzati pesantemente dalla temperatura. Inoltre è apprezzabile
la semplicità del circuito.

\subsection*{Amplificatore alle differenze con sorgente di corrente}

Anche in questo caso abbiamo verificato il corretto funzionamento del circuito, appurando come, con l'aggiunta un sink di corrente costante, sia possibile
risolvere i difetti del circuito base. Il circuito in questione, oltre a risentire poco di effetti termici ha anche un ottimo CMRR e si comporta in modo molto
vicino all'ideale.
