\section*{Dati e risultati}

\subsection*{Amplificatore alle diffrenze}

Lo scopo di questa prima parte è quello di montare un amplificatore alle diffrenze con le seguenti caratteristiche: la corrente di quiescenza $I^q$ deve valere $\SI{0.5}{\milli\ampere}$, un guadagno alle differenze $G\ped{diff} \simeq 30$ e un guadagno in modo comune $G\ped{cm} \simeq 1$.
A tal fine abbiamo posto i seguenti valori per le resistenze: $R\ped{e1} = \SI{120}{\ohm}$, $R\ped{e2} = \SI{120}{\ohm}$, $R\ped{c2} = \SI{10}{\kilo\ohm}$ ed infine $R\ped{1} = \SI{10}{\kilo\ohm}$. Inoltre il circuito è alimentato con una tensione di collettore $V\ped{cc}$ di $\SI{15}{\volt}$ e una tensione di emettitore $V\ped{ee}$ di $\SI{-15}{\volt}$.
Inoltre abbiamo alimentato l'amplificatore con una tensione picco-picco $V\ped{in} = \SI{300}{\milli\volt}$ in entrata dal canale 1.
Quindi a circuito montato andremo a verificarne il funzionamento e ne misureremo il fattore di reiezione a modo comune (CMRR) sia sperimentalmente che teoricamente.
Sperimentalmente si ottiene che:

\begin{equation}
	G\ped{diff.exp}\,=\,\frac{V\ped{out}}{V_2-V_1}\,=\, XXX \qquad G\ped{cm.exp}\,=\,-\frac{V\ped{out}}{V\ped{in}}\,=\, XXX
\end{equation}
\begin{equation}
	CMRR\,=\,\frac{G\ped{diff.exp}}{G\ped{cm.exp}}\,=\, XXX
\end{equation}

\subsection*{Amplificatore alle differenze con sorgente di corrente}