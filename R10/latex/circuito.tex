\section*{Materiale}

La strumentazione e i componenti utilizzati in questa sessione di laboratorio sono i seguenti:

\begin{itemize}
    \setlength{\itemsep}{0pt}
    \item{Breadboard, cavi a banana e cavetti per breadboard;}
    \item{Osilloscopio: Agilent Technologies DSO-X2002A;}
    \item{Generatore di forme d'onde: Agilent Technologies 33120A;}
    \item{Multimetro: Agilent Technologies 34410A;}
    \item{Resistenze varie: \SI{120}{\ohm}, \SI{1.5}{\kilo\ohm}, \SI{10}{\kilo\ohm} e \SI{33}{\kilo\ohm};}
    \item{3 Transistor BC107B;}
\end{itemize}

\section*{Circuito}

\begin{SCfigure}[1][h]
    \small
    \begin{subfigure}[t]{0.38\textwidth}
        \def\svgwidth{\columnwidth}
        \input{circuito1.pdf_tex}
        \caption{Amplificatore differenziale semplice.}
        \label{fig:semplice}
    \end{subfigure}
    \caption{Circuiti costruiti durante l'esperienza. Le resistenze usate nel circuito \ref{fig:semplice} sono state:
        $R_C = R_1 = \SI{10}{\kilo\ohm}$ e $R_E = \SI{120}{\ohm}$.
        Nell'\ref{fig:complesso} si sono usate $R_C = \SI{10}{\kilo\ohm}$, $R_E = \SI{120}{\ohm}$,
        $R_1 = \SI{33}{\kilo\ohm}$, $R_2 = \SI{10}{\kilo\ohm}$ e $R_3 = \SI{1.5}{\kilo\ohm}$}
    \label{fig:circuiti}
    \begin{subfigure}[t]{0.38\textwidth}
        \def\svgwidth{\columnwidth}
        \input{circuito2.pdf_tex}
        \caption{Amplificatore differenziale migliorato con sorgente di corrente costante.}
        \label{fig:complesso}
    \end{subfigure}
\end{SCfigure}
