\section*{Conclusione}

\subsection*{Sorgente di corrente costante}

Grazie al grafico illustrato in Figura \ref{fig:curr} abbiamo verificato che il circuito realizzato si comporta come una sorgente di corrente costante per valori di $R_c$ che appartengono all'intervallo tra $\SI{0}{\kilo\ohm}$ e $\SI{8}{\kilo\ohm}$.

Siamo anche riusciti a spiegare l'andamento della corrente nella regione dove essa non è costante. Possiamo
quindi dire di avere una buona comprensione del funzionamento del circuito.

\subsection*{Amplificatore ad emettitore comune}

Riguardo al circuito \ref{fig:ampli}, non ci resta che discutere del suo corretto funzionamento avvalendoci dell'immagine in Figura \ref{fig:amp}. Come è possibile notare la tensione $V\ped{out}$ è sfasata di 180$^\circ$ e alzata di 5.9 V, valore che avrebbe dovuto essere di $\SI{5.5}{\volt}$.
Il segnale in uscita ($V\ped{out}$) è amplificato di circa 10 volte rispetto al segnale in entrata.
E` possibile anche notare il fenomeno del clipping, infatti $V\ped{out}$ non può superare i 10 V (linea tratteggiata) e non può scendere sotto 1 V più la tensione istantanea della base ($V\ped{in}$). Per questo motivo si vede che nella zona di clipping inferiore la d.d.p. non è costante ma segue il profilo della tensione in ingresso.

\subsection*{Caratteristica in uscita del transistor BC107B}

Per quanto riguarda il circuito in Figura \ref{fig:transistor}, possiamo notare che il grafico riportato in Figura \ref{fig:cara} è un tipico grafico della caratteristica di un transistor BJP. 
Dal grafico si distiguono la regione attiva, quella di saturazione e l'effetto Early. Abbiamo anche calcolato il
guadagno in corrente del transistor, che è risultato fluttuare attorno a 460. Si può notare che il guadagno
aumenta all'aumentare della corrente di base.
