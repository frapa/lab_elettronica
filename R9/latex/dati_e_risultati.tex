\section*{Dati e risultati}

\subsection*{Sorgente di corrente costante}

In questa prima parte dell'esperienza di laboratorio vogliamo dimensionare e studiare il circuito riportato in Figura \ref{fig:}. Il circuito ha una tensione in ingresso ($V\ped{in}$) di $\SI{10}{\volt}$. La resistenza di collettore $R_c$ è variabile, mentre quella di emettitore $R_e$ ha un valore di $\SI{1}{\kilo\ohm}$.
Per dimensionare il circuito, che lavora in un regime di quiescenza, ci siamo serviti della sua proprietà di generare una corrente costate. Ovvero il ramo del circuito relativo al collettore e quello di emettitore saranno percorsi dalla stessa intensità di corrente $\SI{1}{\milli\ampere}$. Quindi conoscendo il valore di $R_e$ sappiamo la differenza di potenziale ai capi dell'emettitore ($V_e$) che vale $\SI{1}{\volt}$, grazie alla legge di Ohm. Inoltre sappiamo che tra base ed emettitore vi è una caduta di tensione di altri $\SI{0.6}{\volt}$, pertanto la differenza di potenziale ai capi di $R_2$ sarà di $\SI{1.6}{\volt}$.
Quindi per ricavare i valori di $R_1$ e $R_2$ ci siamo serviti della nota relazione per un partitore di tensione, overo:

\begin{equation}
	V\ped{out} \,=\, V\ped{in}\cdot\frac{R_2}{R_1+R_2} \qquad\text{da cui si ricava che:}\qquad 8.4\,R_2\,=\,1.6\,R_1
\end{equation}

dove $V\ped{out}$ non è altro che la tensione di base del transistor.
Quindi non ci è rimasto che scegliere un valore molto grande per $R_1$ come $\SI{100}{\kilo\ohm}$ e otteniamo che $R_2$ deve valere circa $\SI{20}{\kilo\ohm}$. Abbiamo scelto un valore di $R_1$ così grande in quanto lo scopo del circuito è quello di convogliare tutta la corrente nel ramo del collettore, pertanto meno corrente passa attraverso $R_1$ e $R_2$ meglio è.

Infine abbiamo acquisito un grafico della corrente di collettore $I_c$ al variare della resistenza di carico $R_c$. I risultati ottenuti sono illustrati nel grafico in Figura \ref{fig:}.

\subsection*{Amplificatore ad emettitore comune}

In questa sezione abbiamo montato e dimensionato il circuito in Figura \ref{fig:}. I requisiti che deve avere il nostro circuito sono quelli di avere una tensione di collettore ($V_0$) di $\SI{10}{\volt}$, una corrente di quiescenza di circa $\SI{1}{\milli\ampere}$ e un guadagno $G$ di $-10$. Inoltre il circuito è alimentato da un segnale in ingresso ($V\ped{in}$) sinusodale di frequenza ($\nu$) $\SI{1}{\kilo\hertz}$.

Noi sappiamo che in regime di quiescenza lungo il ramo collettore-emettitore scorre una corrente di $\SI{1}{\milli\ampere}$ e che $V\ped{out}$ sia uguale a $\frac{V_0}{2}$. Inoltre poichè vogliamo che il guadagno, definito come $G\,=\,\frac{R_c}{R_e}$, sia di $-10$ possiamo ricavare grazie alle leggi di Ohm i valori di $R_c$ e $R_e$. Queste due resistenze hanno rispettivamente valore di $R_c=\SI{5}{\kilo\ohm}$ e $R_e=\SI{0.5}{\kilo\ohm}$.

Infine sfruttando le stesse considerazioni fatte nella sezione prcedente otteniamo i valori di $R_1$ e $R_2$ che sono risultati essere $R_1=\SI{80}{\kilo\ohm}$ e $R_e=\SI{10}{\kilo\ohm}$.
Infine abbiamo determinato il valore di capacità ($C$) sfruttando la relazione che lo lega alla frequenza di risonanza ($\nu$) e  alle due resistenze $R_1$ e $R_2$, ovvero:

\begin{equation}
	C \,=\, \frac{1}{2\,\pi\,\nu\,(R\ped{eq})} \qquad\text{da cui si ricava che:}\qquad C \,\simeq\,\SI{160}{\nano\farad}
\end{equation}

dove $R\ped{eq}$ è il parallelo tra le resistenze $R_1$ e $R_2$.
Sfruttando l'osciloscopio ne abbiamo verificato il funzionamento e il risultato ottenuto è riportato in Figura \ref{fig:}.

\subsection*{Caratteristica in uscita del transistor BC107B}

In questa ultima sezione abbiamo montato il circuito riportato in Figura \ref{fig:}. La resistenza di emettitore $R_e$ ha un valore di $\SI{10}{\ohm}$ mentre la resistenza di base $R_b$ ha un valore di $\SI{100}{\kilo\ohm}$. Abbiamo utilizzato come segnale in ingresso ($V\ped{in}$) un'onda sinusoidale di frequenza ($\nu$) pari a $\SI{1}{\kilo\hertz}$ con una tensione picco-picco di $\SI{10}{\volt}$ ed un offset pari a $+\SI{5}{\volt}$ DC.
Quindi per studiare la caratteristica di uscita del transistor abbiamo utilizzato l'oscilloscopio e abbiamo acquisito tale caratteristica per differenti valori della corrente di base $I_b$ variando la tensione di base ($V_b$). In particolare siamo partiti da una corente minima di base di $\SI{10}{\micro\ampere}$ fino ad arrivare ad una $I_b$ massima di $\SI{90}{\micro\ampere}$ a step di $\SI{10}{\micro\ampere}$.

Il risultato ottenuto è riportato in Figura \ref{fig:}.

Infine abbiamo calcolato per ciascun valore di $I_b$ ilcoefficiente di guadagno in corrente $\beta$ per il valore di tensione di $\SI{5}{\volt}$. Quindi sfruttando la relazione che lega $\beta$ con la corrente di collettore $I_c$ (nota) e la corrente di base $I_b$:

\begin{equation}
	\beta \,=\, \frac{I_c}{I_b}
\end{equation}

abbiamo ottenuto i risultati riportati in Tabella \ref{tab:}.