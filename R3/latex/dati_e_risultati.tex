\section*{Dati e risultati}

\subsection*{Filtro Passa-Basso}

Per ottenere un filtro Passa-Basso, illustrato in figura XXX, abbiamo montato un circuito $R\,C$ con una resistenza in serie ad un condensatore.

Ricordiamo che la frequenza di taglio è quella frequenza che impone ad un segnale in uscita da un circuito, rispetto a quello di entrata, uno sfasamento di $\frac{\pi}{4}$ o in altri termini un'attenuazione della sua intensità di $\frac{1}{\sqrt{2}}$.
Pertanto conoscendo la relazione che sussiste tra la frequenza di taglio ($\nu\ped{0}$), la resistenza ($R$) ed il codensatore ($C$):

\begin{equation}
	\nu\ped{0} \,=\, \frac{1}{2\,\pi\,R\,C}
	\label{eq:taglio basso}
\end{equation}
%
possiamo determinare il valore più appropriato per la capacità ($C$) imponendo un valore di resistenza a piacere e imponendo che la frequanza di taglio del filtro sia di $1\,\si{\kilo\hertz}$. Abbiamo scelto questo valore per la resistenza di taglio al fine di avere una range di frequenze da analizzare che fosse il più omogeneo possibile ed il meglio distribuito, in modo da poter osservare il reale comportamento di tale filtro. (CREDO!!!)

Quindi una volta realizzato il circuito abbiamo raccolto una serie di valori relativi alla tensione in ingresso al circuito ($V\ped{ing}$), la tensione ai capi del condensatore ($V\ped{out}$), e la frequenza del segnale in ingresso ($\nu\ped{i}$), in modo tale da poter studiare la funzione di trasferimento del filtro al variare della frequanza ($\nu\ped{i}$).
I dati ottenuti sono riportati nel seguente grafico:

[GRAFICO]
