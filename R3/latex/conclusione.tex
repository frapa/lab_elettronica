\section*{Conclusione}

Per il filtro Passa-basso possiamo osservare che i dati da noi raccolti risultano essere compatibili con la predizione teorica come illustrato in Figura \ref{fig:g_basso}, inoltre i valori della frequenza di taglio teorica e sperimentale risultano essere compatibili entro le loro incertezze.

Per il filtro Passa-banda possiamo notare che l'andamento sperimentale dell'attenuazione del segnale in funzione della frequanza, ed il relativo sfasamento risultano seguire la predizione teorica fatta eccezione per le basse frequenze. Questo comportamento, come già spiegato, è dovuto al fatto che un'induttore reale possiede anche una propria resistenza e qundi il circuito corretto, che rappresenta al meglio la situazione reale, è riportato in Figura \ref{fig:corr2}. Anche in questo caso le frequenze di risonanza teoriche e sperimentali risultano essere compatibili entro le loro incertezze.

Infine per il filtro a Reiezione di banda, osservando la Figura \ref{fig:g_notch}, possiamo ooservare come i dati sperimentali siano coerenti con la predizione teorica fino a frequenze di circa $100\,\si{\kilo\hertz}$. Infatti, come già segnalato, per avere una predizione teorica in accordo coi dati sperimentali è necessario apportare una modifica al circuito di partenza ed introdurre una capacità in parallelo all'induttanza, come illustrato in Figura \ref{fig:corr}. Anche in questo caso abbiamo che le frequenze di risonanza teorica e sperimentale risultano essere compatibili entro le loro incertezze. 
