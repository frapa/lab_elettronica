\section*{Materiale}

Il materiale utiizzato per questa esperienza di laboratorio è il seguente:

\begin{itemize}
	\item{Decadi di resistenze ($R$), condensatori ($C$) e induttanze ($L$) per le quali l'incertezza sulla misura è stato posto all'$1\,\%$;}
	\item{Osilloscopio: Agilent Technologies in grado di distiguere frequenze di massimo $70\,\si{\mega\hertz}$;}
	\item{Generatore di onde: Agilent Technologies in grado di generare frequenze massime di $15\,\si{\mega\hertz}$;}
\end{itemize}

Inoltre ricordiamo che per le misure prese mediante l'osciloscopio l'incertezza sulle misure è stata posta all'$1\,\%$.

\section*{Circuito}

Durante tutta la sessione di laboratorio la differenza di potenziale ai capi dei nostri circuiti è stata di $1\,\si{\volt}$ pico-pico. Naturalmente si è prestato attenzione che i valori delle resistenze, che saremo andati ad utilizzare, sopportassero l'intensità di corrente associata a tali valori, e quindi non ci fosse il rischio di bruciarle. Inoltre è doveroso specificare che i circuiti sono stati alimentati con forme d'onda sinusoidali.

[IMMAGINI CIRCUITI]



