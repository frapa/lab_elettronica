\section*{Conclusione}

\subsection*{Circuito invertente}

Come è possibile osservare dall'immagine in Figura \ref{fig:g_vs_freq} il guadagno $G$ del circuito realizzato rimane costante fino a frequnze di circa $\SI{10}{\kilo\hertz}$ ed il suo valore è: $G\,=\,XXX$ Quando la frequenza aumenta possiamo notare che $G$ diminuisce. Dal momento che il guadagno è definito come $\frac{V\ped{out}}{V\ped{in}}$ allora, basandoci sui dati raccolti e graficati, possiamo dire che il circuito si comporta come un filtro passa basso. Questa ipotesi è maggiormente giustificabile se si osserva l'andamento dello sfasamento in funzione della frequenza ($\nu$). Infatti, tenendo conto che una delle funzioni del nostro circuito è quella di sfasare di $180^\circ$ i due segnali allora l'attenuazione dello sfasamento corrisponderebbe ad un aumento dello stesso in un filtro passa basso.
Inoltre facciamo notare che è possibile ricavare dal grafico il valore della frequenza di taglio del circuito, che nel nostro caso vale approssiamtivamente $\nu_0\,=\,\SI{}{\kilo\hertz}$.
Detto questo non ci resta che commentare il fenomeno del clamping. Se il circuito fosse ideale allora dovremmo osservare che il valore dell'ampiezza del segnale in uscita non deve essere maggiore di $\SI{30}{\volt}$. Noi abbiamo trovato che, in buona approssimazione l'ampiezza di $V\ped{out}$ ha un valore di circa $\SI{27.3}{\volt}$.   

\subsection*{Circuito non invertente}

Anche in questo caso possiamo osservare, grazie alla Figura \ref{fig:}, che questo circuito, come quello invertente può essere associato ad un filtro passa-basso. La vera diffrenza tra i due circuiti consiste nell'andamento dello sfasamento tra i due seganli ($V\ped{in}$ e $V\ped{out}$) in funzione della frequenza. Infatti in questo caso, come intuibile anche dal nome del circuito, i due segnali dovrebbero essere in fase tra di loro (e lo sono fino a che non si raggiunge la frequenza di taglio che ha un valore di circa $\nu_0\,=\,\SI{}{\kilo\hertz}$), ma dal momento che il circuito può essere un filtro passa-basso lo sfasamento tra $V\ped{in}$ e $V\ped{out}$ aumenta, una volta superata la frequenza di taglio $\nu_0$. Anche per questo circuito èosservabile il fenomeno del camping e come nel caso precedente abbiamo osservato che l'ampiezza massima  del seganle in uscita $V\ped{out}$ vale $\SI{}{\volt}$.  