\section*{Conclusione}

Per quanto concerne al valore della capacità incognita ($C_3$) possiamo notare come il valore teorico (\ref{eq:C_teo_imm}) e il valore sperimentale (\ref{eq:C_exp}) siano compatibili entro le loro incertezze sperimentali. Quindi possiamo affermare che in base ai valori di resistenze e capacità utilizzati il bilanciamento del ponte avviene per i sgueti calori di $R_1$ (resistenza variabile) e di $\nu_0$ (frequenza in ingresso):

\begin{align*}
	\text{Resistenza variabile}& \qquad R_1 \,=\, 1926 \,\pm\, 1 \,\,\si{\ohm} \\
	\text{Frequenza in ingresso}& \qquad \nu_0 \,=\, 9100 \,\pm\, 100 \,\,\si{\hertz} \\
\end{align*}

Per quanto riguarda invece l'utilizzo del ponte di Wien come filtro notch possiamo dire che fa altamente schifo!!! XD.
\newline
\newline
The end
