\section*{Conclusione}

Per quanto concerne al valore della capacità incognita ($C_3$) possiamo notare come il valore teorico (\ref{eq:C_teo_imm}) e il valore sperimentale (\ref{eq:C_exp}) siano compatibili entro le loro incertezze sperimentali. Quindi possiamo affermare che in base ai valori di resistenze e capacità utilizzati il bilanciamento del ponte avviene per i sgueti calori di $R_1$ (resistenza variabile) e di $\nu_0$ (frequenza in ingresso):

\begin{align*}
	\text{Resistenza variabile}& \qquad R_1 \,=\, 1926 \,\pm\, 1 \,\,\si{\ohm} \\
	\text{Frequenza in ingresso}& \qquad \nu_0 \,=\, 9100 \,\pm\, 100 \,\,\si{\hertz} \\
\end{align*}

Per quanto riguarda invece l'utilizzo del ponte di Wien come filtro notch possiamo dire che l'andamento sperimentale concorda con l'andamento teorico, come si evince dal grafico in Figura \ref{fig:notch}. Tuttavia ci sentiamo di dire che la realizzazione di un filtro notch utilizzando un ponte di Wien, non sia la scelta più saggia. Affermiamo ciò poichè l'intervallo centrato attorno alla ``frequenza di risonanza'' ($\nu_0$), del nostro filtro, è molto ampio. Questo fatto quindi pregiudica la risoluzione del filtro che tende a smorzare una vasta gamma di frequenze. In particolare tale intervallo è compreso tra: \SI{200}{\hertz} e \SI{10000}{\hertz}.

Questo è dovuto al fatto che il circuito così realizzato non è un circuito risonante. L'attenuazione e dovuta alla combinazione delle cadute di potenziale ai capi dei due partitori di tensione, che caratterizzano il circuito.
