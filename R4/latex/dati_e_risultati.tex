\section*{Dati e risultati}

\subsection*{Determinazione capacità incognita}

In questa prima parte della relazione illustreremo come abbiamo trovato la capacità incognita ($C\ped{3}$) sfruttando le proprietà del circuito illustrato in Figura \ref{fig:notch}.
Risolvendo questo tipo di circuito, nel caso in cui questo sia bilanciato, noi otteniamo i seguenti risultati:

\begin{equation}
	\frac{R_4}{R_3} \,+\, \frac{C_3}{C_4} \,=\, \frac{R_2}{R_1} \qquad \text{da cui si ottiene} \qquad C_3 \,=\, \Bigl(\frac{R_2}{R_1} \,-\, \frac{R_4}{R_3}\Bigr)\cdot\,C_4
	\label{eq:C}
\end{equation}
\begin{equation}
	\omega^2\,R_3\,R_4\,C_3\,C_4 \,=\, 1 \qquad \text{da cui si ottiene} \qquad C_3 \,=\, \frac{1}{\omega^2\,R_3\,R_4\,C_4}
	\label{eq:C_imm}
\end{equation}
%

In entrambe le equazioni $R_2,\,R_3,\,R_4$ sono i valori delle resistenze, $R_1$ è il valore della resistenza variabile, $C_4$ è il valore della capacità nota mentre $C_3$ quello della capacità incognita ed infine $\omega^2$ non è altro che il valore del periodo dell'impulso, ovvero $\omega \,=\, 2\,\pi\,\nu_0$ dove $\nu_0$ indica la frequenza in ingresso del nostro circuito.

Quindi l'operazione fondamentale da fare al fine di sfruttare le relazioni sopra citate è quella di bilanciare il circuito. Bilanciare il circuito non vuol dire altro che regolare i valori della frequenza in ingresso ($\nu_0$) e della resistenza variabile ($R_1$) in modo tale che la diffrenza di potenziale tra $V_1$ e $V_2$ sia nulla o quantomeno tendente a zero :D.
