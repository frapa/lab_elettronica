\section*{Dati e risultati}

\subsection*{Determinazione della capacità incognita}

In questa prima parte della relazione illustreremo come abbiamo trovato la capacità incognita ($C_3$) sfruttando le proprietà del circuito rappresentato in Figura XXX.
Risolvendo questo tipo di circuito, nel caso in cui questo sia bilanciato, si ottengono i seguenti risultati:

%\begin{equation}
%	\frac{R_4}{R_3} \,+\, \frac{C_3}{C_4} \,=\, \frac{R_2}{R_1} \qquad \text{da cui si ottiene} \qquad C_3 \,=\, %\Bigl(\frac{R_2}{R_1} \,-\, \frac{R_4}{R_3}\Bigr)\cdot\,C_4
%	\label{eq:C}
%\end{equation}
\begin{equation}
	\omega^2\,R_3\,R_4\,C_3\,C_4 \,=\, 1 \qquad \text{da cui si ottiene} \qquad C_3 \,=\, \frac{1}{\omega^2\,R_3\,R_4\,C_4}
	\label{eq:C_imm}
\end{equation}

%In entrambe le equazioni $R_2,\,R_3,\,R_4$ sono i valori delle resistenze, $R_1$ è il valore della resistenza vriabile, $C_4$ è il valore della capacità nota mentre $C_3$ quello della capacità incognita ed infine $\omega^2$ non è altro che il valore del periodo dell'impulso, ovvero $\omega \,=\, 2\,\pi\,\nu_0$ dove $\nu_0$ indica la frequenza in ingresso del nostro circuito.

dove $R_3,\,R_4$ indicano le resistenze da $1\,\si{\kilo\ohm}$, $C_4$ è il valore della capacità nota ed infine $\omega$ non è altro che il valore del periodo dell'impulso, ovvero $\omega \,=\, 2\,\pi\,\nu_0$ dove $\nu_0$ indica la frequenza in ingresso del nostro circuito.

Quindi l'operazione fondamentale da fare al fine di sfruttare la relazione sopra citata è quella di bilanciare il circuito. Bilanciare il circuito non vuol dire altro che regolare i valori della frequenza in ingresso ($\nu_0$) e della resistenza variabile ($R_1$) in modo tale che la diffrenza di potenziale tra $V_2$ e $V_1$ sia nulla o quantomeno tendente a zero :D.

%I valori teorici $C\ped{3teo}$ da noi trovati, sfruttando le due formule precedenti (\ref{eq:C_imm}) e (\ref{eq:C_teo_imm}), per la capacità incognita sono i seguenti:  
Il valore teorico $C\ped{3teo}$ da noi trovato per la capacità incognita è il seguente:

%\begin{equation*}
%	C\ped{3teo} \,=\, XXX \qquad \text{sfruttando la rezione (\ref{eq:C})}
%\end{equation*}
\begin{equation}
	C\ped{3teo} \,=\, 2.9 \,\pm\, 0.2 \,\si{\nano\farad}  \qquad \text{sfruttando la rezione (\ref{eq:C_imm})}
	\label{eq:C_teo_imm}
\end{equation}
%2.972638538603707e-09
%2.1155093730419337e-10

Mentre il valore sperimentale di $C\ped{3exp}$ ottenuto grazie alla misura diretta col multimetro digitale è:

\begin{equation}
	C\ped{3exp} \,=\, 3.0 \,\pm\, 0.2 \,\si{\nano\farad}
	\label{eq:C_exp} 
\end{equation}

\subsection*{Ponte di Wien come filtro notch}

In questa seconda sezione della relazione vogliamo scoprire se sia possibie utilizzare il ponte di Wien come filtro notch. A tal fine è necessario apportate alcune modifiche strutturali al nostro circuito. Tali cambiamenti sono illustrati in Figura XXX.

Al fine di verificare se il nostro circuito sia approssimabile ad un filtro notch dobbiamo studiare la funzione di trasferimento del circuito ($\frac{V\ped{out}}{V\ped{in}}$) al variare della frequenza del segnale in entrata ($\nu_i$) mediante i diagrammi di Bode. Ricordiamo inoltre che la diffrenza di potenziale ai capi del nostro circuito ($V\ped{in}$) è rimasta costante per tutta la durata dell'esperienza. $V\ped{out}$ è il segnale in uscita dal nostro circuito.

I dati da noi ottenuti sono riportati nei seguenti grafici:

[GRAFICI]