\section*{Materiale}

Il materiale utilizzato per questa esperienza di laboratorio è il seguente:

\begin{itemize}
	\item{Resistenze ($R$) da $1\,\si{\kilo\ohm}$ con un incertezza percentuale del $5\,\%$, e una resistenza variabile con un'incertezza nominale dell'$1\,\%$;}
	\item{Condensatori ($C$);}
	\item{Osilloscopio: Agilent Technologies in grado di distiguere frequenze di massimo $70\,\si{\mega\hertz}$;}
	\item{Generatore di onde: Agilent Technologies in grado di generare frequenze massime di $15\,\si{\mega\hertz}$;}
	\item{Multimetro: Agilent Technologies;}
\end{itemize}

L'errore che abbiamo attribuito alle misure prese mediante il multimero e l'oscilloscopio è quello di risoluzione dello strumento, che a seconda del tipo di misura assume un valore differente.

\section*{Circuito}

I circuiti utilizzati sono i seguenti:

[IMMAGINE]

Ricordiamo che per tutta la durata dell'esperienza la differenza di potenziale utilizzata è stata di $5\,\si{\volt}$ pico-pico, inoltrela forma d'onda in ingresso al nostro circuito era sinusoidale.