\section*{Materiale}

Il materiale utilizzato per questa esperienza di laboratorio è il seguente:

\begin{itemize}
    \setlength{\itemsep}{1pt}
	\item{Resistenze ($R$) da $1\,\si{\kilo\ohm}$ con un incertezza percentuale del $5\,\%$, e una resistenza variabile con un'incertezza nominale dell'$1\,\%$;}
	\item{Condensatori ($C$);}
	\item{Osilloscopio: Agilent Technologies in grado di distiguere frequenze di massimo $70\,\si{\mega\hertz}$;}
	\item{Generatore di onde: Agilent Technologies in grado di generare frequenze massime di $15\,\si{\mega\hertz}$;}
	\item{Multimetro: Agilent Technologies;}
\end{itemize}

L'errore che abbiamo attribuito alle misure prese mediante il multimero e l'oscilloscopio è quello di risoluzione dello strumento, che a seconda del tipo di misura assume un valore differente.

\section*{Circuito}

\begin{wrapfigure}{r}{0.5\textwidth}
  \centering
  \vspace{-1cm}
  \begin{circuitikz}[scale=0.8, transform shape, font=\Large] \draw
   (0,0) 
    to [short, o-*] (3, 0)
    to [short, *-] (3, 0.5) -- (2, 0.5)
    to [R, l=$R_3$] (2, 2.5) -- (4, 2.5)
    to [C, l=$C_3$] (4, 0.5) -- (3, 0.5)
    (3, 2.5) to [short, -*] (3, 3)
    to [short, *-o] (5.5, 3) node[anchor=west]{$V_1$}
    (3, 6) to [vR, l=$R_1$, -*] (3, 3)
    (3, 6) to [short, *-o] (0, 6)
    (3, 0) -- (10, 0)
    to [C, l=$C_4$] (10, 1)
    to [R, l=$R_4$] (10, 3)
    to [short, *-o] (8, 3) node[anchor=east]{$V_2$}
    (10, 3) to [R, l=$R_2$] (10, 6) -- (2, 6);
    \draw[triangle 90-triangle 90] (0, 0.3) -- (0, 5.7);
    \node[anchor=west] at (0, 3) {$V\ped{in}$};
  \end{circuitikz}
  \caption{Ponte di Wien utilizzato per ricavare $C_3$.}
  \label{fig:r}
  \vspace{-1cm}
\end{wrapfigure}

Ricordiamo che per tutta la durata dell'esperienza la differenza di potenziale utilizzata è stata di $5\,\si{\volt}$ pico-pico, inoltre la forma d'onda in ingresso al nostro circuito è sinusoidale.
