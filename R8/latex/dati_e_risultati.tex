\section*{Dati e risultati}

\subsection{Transistor come interruttore}

In questa sezione vogliamo verificare il corretto funzionamento del circuito illustrato in Figura \ref{fig:}.
Il circuito comprende un transistor BC107B, un diodo LED, una resistenza $R_c=1\,\si{\kilo\ohm}$ e una resistenza variabile $R_b=100\,\si{\kilo\ohm}$. Il circuito è stato alimentato con una tensione costante in ingresso $V\ped{in}$ di $15\,\si{\volt}$. Inoltre abbiamo prestato attenzione che la corrente passante per il LED non superasse mai i $15\,\si{\milli\ampere}$.

Abbiamo verificato che il transistor si comportasse effettivamente come un interruttore agendo manualmente sul cavo banana-banana che collega $R_b$ con il multimetro digitale. Ovvero abbiamo appurato che, scollegando $R_b$ dal cavo, il LED non si illumina mentre, se $R_b$ e il cavo sono collegati, allora il LED è illuminato.

Successivamente abbiamo collegato un amperometro tra $R_b$ e il terminale di base del transistor. In questo modo siamo in grado di osservare come varia la corrente di collettore $I_c$ al variare della resistenza $R_b$. Questo è possibile in quanto il multimetro ci fornisce il valore della corrente di emettitore $I_e$, pertanto sottraendo ad $I_e$ il valore corrispondente di $I_b$ otteniamo il valore della corrente di collettore $I_c$. Inoltre, grazie a queste misure, possiamo determinare come varia la corrente di base $I_b$ sempre al variare di $R_b$.

I risultati ottenuti sono riportati nei grafici in Figura \ref{fig:} e Figura \ref{fig:}.

\subsection{Transistor come interruttore veloce}

In questa sezione vogliamo verificare che il transistor, utilizzato nel circuito riportato in Figura \ref{fig:}, funzioni effettivamente come interruttore veloce.
Come nel caso precedente la tensione in ingresso $V_0$ è costante ed ha un'intensità di $15\,\si{\volt}$. La resistenza $R_c$ ha un valore di $1\,\si{\kilo\ohm}$ mentre la resistenza $R_b$ vale $10\,\si{\kilo\ohm}$. Inoltre per pilotare il terminale di base del transistor vi abbiamo applicato un segnale in ingresso TTL $V\ped{in}$ fornitoci dal generatore di forme d'onda sotto forma di onda quadra di intervallo $0-5\,\si{\volt}$.

Come abbiamo verificato il transistor utilizzato in questo circuito è effettivamente utilizzabile come interruttore veloce. Infatti fino ad una frequenza massima di circa $20\,\si{\kilo\hertz}$ era possibile distinguere chiaramenti gli intervalli in cui il LED si accendeva e si spegneva. Inoltre per frequenze superiori ai $20\,\si{\kilo\hertz}$  si sarebbe potuto verificare il corretto funzionamento del transistor, come interruttre rapido, utilizzando un fotodiodo[???] con cui avremmo appurato che tale componente è utilizzabile anche con frequenze di circa $15\,\si{\mega\hertz}$, range massimo di funzionamento del nostro generatore di onde.
Il motivo per cui il LED si accende e si spegne è facilmente spiegabile in quanto: [A cosa è dovuto]

\subsection{Emitter follower}

\subsection{}

\subsection{}
