\section*{Dati e risultati}

\subsection{Transistor come interruttore}

In questa sezione vogliamo verificare il corretto funzionamento del circuito illustrato in Figura \ref{fig:}.
Il circuito comprende un transistor BC107B, un diodo LED, una resistenza $R_c=1\,\si{\kilo\ohm}$ e una resistenza variabile $R_b=100\,\si{\kilo\ohm}$. Il circuito è stato alimentato con una tensione costante in ingresso $V\ped{in}$ di $15\,\si{\volt}$.
Abbiamo verificato che il transistor si comportasse effettivamente come un interruttore agendo manualmente sul cavo banana-banana che collega $R_b$ con il multimetro digitale. Ovvero abbiamo appurato che, scollegando $R_b$ dal cavo, il LED non si illumina mentre, se $R_b$ e il cavo sono collegati, allora il LED è illuminato.

\subsection{}

\subsection{}

\subsection{}

\subsection{}
