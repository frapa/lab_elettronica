\section*{Conclusione}

\subsection*{Emitter follower}

Come è possibile osservare dalla Figura \ref{fig:} notiamo che il circuito (c) ci permette di visualizzate in output solamente la parte di segnale dovuta alla semionda positiva, mentre la parte negativa del segnale viene completamente tagiata. Questo comportamento è analogo a quello visto nel punto precedente. Ovvero la condizione necessaria affinche passi corrente tra i terminali di base e emettitore del ransistor è che devono essere polarizzati in diretta. Dal momento che la base del transistor si trova ad una tensione di $0\,\si{\volt}$ allora base ed emettitore si trovano in condizione di polarizzazione diretta soltanto quando il segnale in ingersso alla base ($V\ped{in}$) è positivo. Al contrario (per $V\ped{in}\,\leq\,0\,\si{\volt}$ ) non abbiamo passaggio di corrente e pertanto $V\ped{out}$ risulta nullo.
