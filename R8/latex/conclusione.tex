\section*{Conclusione}

\subsection*{Transistor come interruttore veloce}

Il motivo per cui il LED si accende e si spegne è facilmente spiegabile in quanto: il terminale emettitore del transistor si trova ad un potenziale di $0\,\si{\volt}$. Quindi affinche possa passare corrente tra la base e l'emettitore occorre che tra queste vi sia una differenza di potenziale di almeno $0.6\,\si{\volt}$. Quindi dal momento che il segnale in ingresso permette sostanzialmente due valori di tensione di base ($V_b$), ovvero 0 e 5 $\si{\volt}$, allora quando $V_b=0\,\si{\volt}$ il ramo base-emettitore non conduce e nel circuito non passa corrente. Al contrario quando $V_b=5\,\si{\volt}$, allora il ramo base-emettitore entra in conduzione e nel circuito passa corrente e pertanto il LED si illumina.

\subsection*{Emitter follower}

Come è possibile osservare dalla Figura \ref{fig:} notiamo che il circuito (c) ci permette di visualizzate in output solamente la parte di segnale dovuta alla semionda positiva, mentre la parte negativa del segnale viene completamente tagiata. Questo comportamento è analogo a quello visto nel punto precedente. Ovvero la condizione necessaria affinche passi corrente tra i terminali di base e emettitore del ransistor è che devono essere polarizzati in diretta. Dal momento che la base del transistor si trova ad una tensione di $0\,\si{\volt}$ allora base ed emettitore si trovano in condizione di polarizzazione diretta soltanto quando il segnale in ingersso alla base ($V\ped{in}$) è positivo. Al contrario (per $V\ped{in}\,\leq\,0\,\si{\volt}$ ) non abbiamo passaggio di corrente e pertanto $V\ped{out}$ risulta nullo.

\subsection*{Emitter follower polarizzato}

Come illustrato in Figura \ref{fig:} in questa configurazione il nostro circuito permette di visualizzare in output ($V\ped{out}$) un segnale che è identico a quello in ingresso. Questo risultato differisce da quello ottenuto per un semplice emitter follower (analizzato nella sezione precedente), in quanto l'emettitore è ad uan tensione $V_e$ di $-12\,\si{\volt}$. Questo comporta che, quando la base si trova ad una tensione $V\ped{in}\,\leq\,0\,\si{\volt}$, base ed emettitore risultano essere comunque polarizzati in diretta. Pertanto peremettono un passaggio di corrente anche quando il segnale è negativo.

Inoltre abbiamo osservato come si comporta $V\ped{out}$ al variare di $V\ped{in}$. Quello che abbiamo ottenuto è che... . Questo fenomeno è detto clamping.
